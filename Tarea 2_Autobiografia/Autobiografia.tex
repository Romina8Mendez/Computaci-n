\documentclass[letterpaper, 12pt]{article}
\usepackage[utf8]{inputenc}
\usepackage[spanish]{babel}
\usepackage[dvipsnames]{xcolor}

\title{Datos que cuentan un pedacito de mi vida. }
\author{Méndez González Romina Estephania}
\date{11 de septiembre de 2022}

\begin{document}

\maketitle

\section{\Large{Academia}}
    \subsection{\large{Pasado:}}
        \normalsize{Soy egresada de la Escuela Nacional Preparatoria No. 8 ``Miguel E. Schulz''. En realidad, la preparatoria me quedaba relativamente cerca; sin embargo, el tráfico hacía que pareciera lo contrario. Me tardaba aprox. una hora con treinta minutos para llegar. La ruta que tomaba era la más rápida y consistía en tomar un taxi que me dejaba en Santa Fe, de ahí tomaba un camión, el cuál, me dejaba en Periférico y finalmente, tomaba un camión que me dejaba cerca de la prepa.}  
    \subsection{\large{Actualidad:}}
        \normalsize{La Facultad no queda cerca de mi domicilio, ya que me hago dos horas (o incluso, dos horas y media). En general, la ruta que hago consiste en tomar un camión que me deja en San Antonio, ahí tomo el metro a Barranca del Muerto y saliendo de la estación, tomo un camión que me deja en el Estadio de C.U, posteriormente espero por muuuucho tiempo el pumabus que me deja cerca de la Facultad de Ciencias y finalmente, camino unos cinco minutos hasta la Fac.}
    \subsection{\large{¿Por qué física?}}
        \normalsize{Verdaderamente no sé que me llevó a decidir estudiar esta carrera. Tal vez porque ello implicó muchas vivencias y experiencias del pasado. Algunas de estas vivencias nacen desde la secundaria con la materia de Física, después con los vídeos de divulgación científica y mis ganas de aprender se expandieron cuando entré a la prepa y principalmente, cuando participé en la Olimpiada Universitaria del Conocimiento. Supongo que influyeron más cosas en esta decisión, pero las que mencioné son las más importantes :)}
    

\section{\Large{Hobbies o pasatiempos}}
    \subsection{\large{Ver videos }}
        \normalsize{Esto no es un hobbie o algo así, es solo para distraer a mi mente del trayecto y ajetreo de la escuela. Le dedico no más de una seis horas a la semana porque luego tengo que hacer tarea :(}
    \subsection{\large{Ver la televisión}}
        \normalsize{Este solo lo hago los fines de semana y de vez en cuando entre semana. La veo en la noche y no pasa de 4 horas que miro TV a la semana. Me gusta porque a veces hay programas de tele muy cool, como Survivor o Master Cheff.}


\section{\Large{Música preferida}}
    \subsection{\large{Indie}}
        \normalsize{Este género lo escucho a diario, es el pan de cada día; aunque depende como me sienta, pues pongo ciertas canciones más o menos profundas u.u  Algunos exponentes de este género es mi diosa Lana del Rey con la canción de ``you can be the boss'' y The Neightbourhood con la canción ``Flawless''.} 
    \subsection{\large{Hip-Hop/Rap}}
        \normalsize{Este género no lo escucho siempre, pero lo pongo siempre que tengo un felling diferente... es difícil explicar. Pero los artistas que destancan son G-eazy con la canción ``The Beautiful \& Damned'' y Blackbear con ``Hot girl bummer''.}
    
    
\section{\Large{Cosas que me gustan}}
    \subsection{\large{Mi crush}}
        \normalsize{Verdaderamente es una persona guapísima, inteligente y con una un aura muy bomnita y relax.}
    \subsection{\large{Salir a pasear}}
        \normalsize{Me gusta mucho salir con mis amig@s a plazas, al centro de la cdmx, a comer, a comprar cosas random y sobre todo, caminar en parques bonitos por la noche :3}
    \subsection{\large{Hacerme tatuajes}}
        \normalsize{Está genial cuando no duelen, pero dejando de lado eso, siento que se ven muy estéticos y enmarcan una parte muy profunda de tus pensamientos o de tu personalidad.}
    
    
\section*{Puntos extras}
1. Ingresa dos párrafos con la información que más te agrade con las siguientes características:\\


El primero debe contener formato itálica y formato negritas:\\ 
    \textit{ \textbf{ -¡Ah, de qué cosa más insignificante depende la felicidad! -lloraba el joven-. El príncipe da un baile mañana por la noche y mi amada asistirá a la fiesta. Si le llevo una rosa roja, bailará conmigo y la tendré en mis brazos. Pero no hay rosas rojas en mi jardín, así que la perderé para siempre.}}\\ 
    
El segundo debe estar en dos distintos tipos de color de letra:\\
    \textcolor{BrickRed}{-Sé feliz -le dijo el ruiseñor-, tendrás tu rosa roja. La crearé con notas de música al claro de luna} \textcolor{SeaGreen}{y la teñiré con la sangre de mi propio corazón. Lo único que te pido es que seas un verdadero enamorado.}
 
        
    
\end{document}
